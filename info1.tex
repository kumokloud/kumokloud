%ドキュメントクラスの指定。使えるならばjsarticleを指定し、
% そうでなければjarticleを指定する。しかしjsarticleがない環境は
% 新しい環境に更新すべきだと思う。
%   - 10pt:       本文の文字サイズを10ptにする。実用上は10ptか11ptを指定する。
%   - a4j:        用紙サイズを日本語用A4サイズにする。
%                 報告書は日本語なので必要。
%   - twocolumn:  二段組にする。
%                 報告書のフォーマットは二段組なので基本的に必要。
%   - dvipdfmx:   画像の描画にdvipdfmxを使う。
%                 後述のgraphicxパッケージを使うときに必要。
%   - uplatex:    TeXの処理系をupLaTeXにする。
%                 **処理系がpLaTeXのときは 削 除 すること。**
\documentclass[10pt, a4j, uplatex, dvipdfmx, twocolumn]{jsarticle}


\usepackage{graphicx}   % \includegraphics を使うためのパッケージ読み込み
\usepackage{imcreport}  % IMC報告会用テンプレートパッケージの読み込み


% ハイフネーションを抑制する。数値が大きいほど、無理にでも抑制しようとする。
\hyphenpenalty=10000\relax
\exhyphenpenalty=10000\relax
\sloppy


\title{機械学習の強化学習}   % タイトル
\author{瀬戸川大揮}                                % 著者
\studentnumber{TB16K049}                        % 学籍番号
\date{2019年5月10日}                            % 日付
\header{中間報告}                 % 左上のヘッダ内容


\begin{document}

\maketitle

\section{はじめに}

昨今画像認識や将棋電脳戦で注目を集めている機械学習には種類があり、教師あり学習、教師なし学習、強化学習の3つに種別される。その中の強化学習について学習したので報告する。

\section{強化学習}

強化学習は答えを教えるのではなく、目標に達した際にそれまでの過程を評価し学習させていくものである。
学習において、行動を行う主体をエージェントとし、働きかけられるものを環境とする。エージェントが環境に対して働きかけることを行動と呼び、環境は行動によって変化する。その状態での行動の良し悪しは報酬というもので決まる。そして、エージェントは環境と報酬を元に行動を決定する。このような一連な流れ

\section{図の挿入}

図~\ref{バッテンのラベル}のように、\pLaTeX では図を挿入することができる。
\begin{figure}[bp]
  \centering
  \includegraphics[width=0.45\textwidth]{./sample_figure.pdf}
  \caption{バッテン}
  \label{バッテンのラベル}
\end{figure}

\section{表の挿入}

\pLaTeX で作った表を表~\ref{表のラベル}に示す。
\begin{table}[bp]
  \centering
  \caption{表のタイトル}
  \label{表のラベル}
  \begin{tabular}{c|cccc}
    \hline
          & 項目A & 項目B & 項目C & 項目D \\
    \hline \hline
    項目a & 1     & 2     & 3     & 4 \\
    項目b & 5     & 6     & 7     & 8 \\
    \hline
  \end{tabular}
\end{table}

空間を埋めるためだけの文章です.空間を埋めるためだけの文章です.空間を埋めるためだけの文章です.空間を埋めるためだけの文章です.空間を埋めるためだけの文章です.空間を埋めるためだけの文章です.空間を埋めるためだけの文章です.空間を埋めるためだけの文章です.空間を埋めるためだけの文章です.空間を埋めるためだけの文章です.空間を埋めるためだけの文章です.空間を埋めるためだけの文章です.空間を埋めるためだけの文章です.空間を埋めるためだけの文章です.空間を埋めるためだけの文章です.空間を埋めるためだけの文章です.空間を埋めるためだけの文章です.空間を埋めるためだけの文章です.空間を埋めるためだけの文章です.空間を埋めるためだけの文章です.空間を埋めるためだけの文章です.

\subsection{サブセク}

空間を埋めるためだけの文章です.空間を埋めるためだけの文章です.空間を埋めるためだけの文章です.空間を埋めるためだけの文章です.空間を埋めるためだけの文章です.空間を埋めるためだけの文章です.空間を埋めるためだけの文章です.空間を埋めるためだけの文章です.

\end{document}

